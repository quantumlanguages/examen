% type
\documentclass{article}

% page format
\usepackage[letterpaper, margin=1.5cm]{geometry}
\usepackage[utf8]{inputenc}

% font
\renewcommand{\familydefault}{\sfdefault}

% colors
\usepackage[dvipsnames]{xcolor}

% enumeration format
\usepackage{enumitem}
\setlist[enumerate,1]{label=\textcolor{RoyalPurple}{\arabic*.}}
\setlist[enumerate,2]{label=\textcolor{RoyalPurple}{\alph*)}}
\setlist[itemize,1]{label=\color{RoyalPurple}{$\bullet$}}

% language
\usepackage[spanish]{babel}

% math symbols
\usepackage{amssymb}
\usepackage{amsmath}

% code
\usepackage{minted}
\usemintedstyle{manni}

% shortcuts
\newcommand{\tx}[1]{\textcolor{Black} {\texttt{#1}}}
\newcommand{\tp}[1]{\textcolor{CadetBlue} {\texttt{#1}}}
\newcommand{\tb}[1]{\textcolor{RoyalPurple} {\textbf{#1}}}
\newcommand{\ti}[1]{\textcolor{RoyalPurple} {\textit{#1}}}
\newcommand{\pt}[1]{\textcolor{RoyalPurple}{(#1pt)}}
\newcommand{\tvec}[1]{$\vec{\tx{#1}}$ }
\newcommand{\es}{$\square$}
\newcommand{\pop}[2]{ \tx{#1} \succ \tx{#2}}
\newcommand{\push}[2]{ \tx{#1} \prec \tx{#2}}
\newcommand{\prop}[2]{\tx{#1} \ll \tx{#2}}
\newcommand{\kr}{\rightarrow_{\mathcal{K}} \quad}
\newcommand{\br}{\rightarrow_{\beta} \quad}
\newcommand{\krs}{\rightarrow_{\mathcal{K}}^{\star} \quad}


% header
\title {
    \color{RoyalPurple}
    Lenguajes de Programación 2020-1\\
    Facultad de Ciencias UNAM\\
    Tarea Examen Parcial 4
}

\author{
    Sandra del Mar Soto Corderi \quad Edgar Quiroz Castañeda
}

\date{
    Fecha de entrega: 2 de diciembre de 2019
}

% document
\begin{document}
    % header
    \maketitle

    \tb{Esta tarea vale 8 puntos sobre el parcial 4, la calificación se
    completa con una pregunta presencial el 29 de noviembre de 2019}

    \begin{enumerate}
        \item \pt{1} Se desea implementar una función \tp{ct} que reciba un
        árbol heterogéneo de naturales o booleanos y devuelva la conjunción de
        sus elementos siempre y cuando todos sean booleanos y en otro caso
        devuelva el valor \tp{n+1}, donde \tp{n} es el primer natural encontrado en el
        árbol. Para este propósito defina una función \tp{ctaux} y una expresión
        \tp{e} tal que la función \tp{ct} quede implementada como

        \begin{minted}{haskell}
ct t = handle ctaux t with x => e
        \end{minted}

        Bosqueje la evaluación de la expresión

        \begin{minted}{haskell}
Node (iszero 9) (Node False Void Void) (Node 5 Void Void)
        \end{minted}

        en la máquina \tp{$\mathcal{K}$}.

        Puede omitir varios pasos pero no los que involucren el manejo de excepciones.

        \ti{Sugerencia:} Es más fácil si define \tp{ctaux} a partir de una
        función binaria \tp{vand} que realice la conjunción si sus argumentos
        son booleanos y en caso contrario lance una excepción adecuada de forma
        que sea manejada por \tp{e}. Puede suponer que existe una función unaria
        \tp{isbool} que verifica si su argumento es o no un booleano.

        \tb{Solución:}

        Definiendo \tp{vand}.

        \begin{minted}{haskell}
-- Función auxiliar para mandar excepciones
bool_panic p = let x = p in if isbool x then x else raise(x)

-- Realiza la conjunción si sus argumentos son booleanos o lanza una
-- excepción con el primer argumeto que no lo sea.
vand p q = and (bool_panic p) (bool_panic q)
        \end{minted}

        Ahora, hay que usar esta función para definir \tp{ctaux}.

        \begin{minted}{haskell}
ctaux Void = True
ctaux (Node r i d) = vand r (vand (ctaux i) (ctaux d))
        \end{minted}

        Y usando esto, se define \tp{ct} como

        \begin{minted}{haskell}
ct t = handle ctaux t with x => x + 1
        \end{minted}

<<<<<<< HEAD
        Se definen algunos sinónimos para hacer más corta la evaluación

        \begin{minted}{haskell}
-- constantes
f = False
t = True
-- arboles
i = (Node False Void Void)
d = (Node 5 Void Void)
a = (Node (iszero 9) i d)
-- expresiones booleanas simples
p1 = iszero 9
-- pilas
s1 = and(-, e2), handle(-, x.x+1)
s2 = and(f, -), handle(-, x.x+1)
s3 = and(-, e7), let(-, x.e10), s2
s4 = and(f, -), let(-, e10), s3
-- otras expresiones
e3 = (vand (ctaux i) (ctaux d))
e2 = bool-panic e3
e4 = ctaux i
e5 = (bool-panic (ctaux i))
e6 = ctaux d
e7 = (bool-panic (ctaux d))
e11 = ctaux Void
e12 = bool-panic e11
e8 = (vand e11 e11)
e9 = bool-panic e8
e10 = if (isbool x) then x else raise(x)
        \end{minted}

        Ahora, vamos a evaluar la expresión:
=======
>>>>>>> sandy

		Ahora, para evaluar la expresión, es útil definir lo siguiente
		
		
		\begin{align*}
		&\tx{a = Node (iszero 9) (Node False Void Void) (Node 5 Void Void)} \\
		&\tx{r = iszero 9}\\
		&\tx{i = Node False Void Void}\\
		&\tx{d = Node 5 Void Void}
		\end{align*}

        Evaluamos:
        
        \begin{align*}
<<<<<<< HEAD
        % evaluando a
        &\pop{\es}{ct a}\\
        &\br \pop{\es}{handle ctaux a with x => x+1}\\
        &\kr \pop{handle(-, x.x+1)}{ctaux a}\\
        &\br \pop{handle(-, x.x+1)}{and (bool-panic p1) e2}\\
        &\kr \pop{and(-, e2), handle(-, x.x+1)}{bool-panic p1}\\
        &\br \pop{s1}{let x = p1 in e10}\\
        &\kr \pop{let(-, x.e10), s1}{iszero 9}\\
        &\krs \push{let(-, e10), s1}{f}\\
        &\kr \pop{s1}{if isbool x then x else raise(x) [x := f]}\\
        &\krs \push{and(-, e2), handle(-, x.x+1)}{f}\\
        % evaluando los hijos
        &\kr \pop{and(f, -), handle(-, x.x+1)}{bool-panic e3}\\
        &\br \pop{s2}{let x = e3 in e10}\\
        &\kr \pop{let(-, x.e10), s2}{e3}\\
        &\br \pop{let(-, x.e10), s2}{and e5 e7}\\
        % evaluando el subárbol derecho
        &\kr \pop{and(-, e7), let(-, x.e10), s2}{e5}\\
        &\br \pop{s3}{let x = e4 in e10}\\
        &\kr \pop{let(-, e10), s3}{e4}\\
        % evaluando raíz del subárbol derecho
        &\br \pop{let(-, e10), s3}{and (bool-panic f) e9}\\
        &\kr \pop{and(-, e9), let(-, e10), s3}{bool-panic f}\\
        &\br \pop{and(-, e9), let(-, e10), s3}{let x = f in e10}\\
        &\krs \push{and(-, e9), let(-, e10), s3}{f}\\
        % evaluando subárboles del subárbol derecho
        &\kr \pop{and(f, -), let(-, e10), s3}{e9}\\
        &\br \pop{s4}{let x = e8 in e10}\\
        &\kr \pop{let(-, x.e10), s4}{e8}\\
        &\br \pop{let(-, x.e10), s4}{and e12 e12}\\
        &\kr \pop{and(-, e12), let(-, x.e10), s4}{e12}\\
        &\br \pop{and(-, e12), let(-, x.e10), s4}{let x = e11 in e10}\\
    \end{align*}
    \begin{align*}
        &\kr \pop{let(, x.e10), and(-, e12), let(-, x.e10), s4}{ctaux Void}\\
        &\kr \push{let(, x.e10), and(-, e12), let(-, x.e10), s4}{t}\\
        &\kr \pop{and(-, e12), let(-, x.e10), s4}{(if isbool x then x else raise(x))[x := true]}\\
        &\krs \pop{and(true, -), let(-, x.e10), s4}{e12}\\
        &\krs \push{and(true, -), let(-, x.e10), s4}{t}\\
        &\krs \pop{s4}{(if isbool x then x else raise(x))[x := t]}\\
        &\krs \push{and(f, -), let(-, e10), s3}{t}\\
        &\krs \pop{s3}{(if isbool x then x else raise(x))[x := f]}\\
        &\krs \push{and(-, e7), let(-, x.e10), s2}{f}\\
        &\krs \pop{and(f, -), let(-, x.e10), s2}{e7}\\
        &\br \pop{and(f, -), let(-, x.e10), s2}{let x = e6 in e10}\\
        &\kr \pop{let(-, x.e10), and(f, -), let(-, x.e10), s2}{e6}\\
        &\br \pop{let(-, x.e10), and(f, -), let(-, x.e10), s2}{and (bool-panic 5) e8}\\
        &\kr \pop{and(-, e8), let(-, x.e10), and(f, -), let(-, x.e10), s2}{bool-panic 5}\\
        &\br \pop{and(-, e8), let(-, x.e10), and(f, -), let(-, x.e10), s2}{let x = 5 in e10}\\
        &\krs \pop{and(-, e8), let(-, x.e10), and(f, -), let(-, x.e10), s2}{if isbool 5 then 5 else raise(5)}\\
        &\krs \prop{and(-, e8), let(-, x.e10), and(f, -), let(-, x.e10), s2}{raise(5)}\\
        &\krs \prop{and(f, -), let(-, x.e10), s2}{raise(5)}\\
        &\krs \prop{and(f, -), handle(-, x.x+1)}{raise(5)}\\
        &\krs \prop{handle(-, x.x+1)}{raise(5)}\\
        &\krs \pop{\es}{(x+1)[x := 5]}\\
        &\krs \push{\es}{6}
=======
        &\pop{\es}{ct a} \\
        &\br \pop{\es}{handle ctaux a with x => x + 1}\\
        &\kr \pop{handle(-, x.x+1)}{ctaux a}\\
        &\br \pop{handle(-, x.x+1)}{vand r (vand (ctaux i) (ctaux d))}\\
        &\br \pop{handle(-, x.x+1)}{and bool\_panic r (bool\_panic (vand (ctaux i) (ctaux d)))}\\
        &\kr \pop{and(-,(bool\_panic (vand (ctaux i) (ctaux d))));handle(-, x.x+1)}{and bool\_panic r }\\
>>>>>>> sandy
        \end{align*}

        \item \pt{1} Considere el siguiente programa, donde suponemos que el
        lenguaje contiene un operador primitivo \tp{not} para la negación
        booleana.

        \begin{minted}{haskell}
not letcc k2 in
    iszero(2 + letcc k1 in
        3  + if x = pred 8 then
<<<<<<< HEAD
            4 * pred (continue k1 6
            else 5 * suc (continue k2 false)
        end
        )
=======
            	4 * pred (continue k1 6)
             else 5 * suc (continue k2 false)
          end
    )
>>>>>>> sandy
    end
end
        \end{minted}
        \begin{enumerate}
            \item ¿Cuáles son los tipos de \tp{k1} y \tp{k2}?

            \[
                \tx{k1 : } \tp{Cont(Integer)} \quad \tx{k2 : } \tp{Cont(Bool)}
            \]

            \item ¿A qué continuaciones se ligan las variables \tp{k1} y \tp{k2}?

<<<<<<< HEAD
            \[
                \tp{k1 --> (continue k1 6)} 
                \kr \tp{(continue (not(-), iszero(-), suma(2, -)) 6)}
            \]

            \[
                \tp{k2 --> (continue k2 false)} 
                \kr \tp{(continue (not(-)) false)}
            \]

=======
			\[
			\tx{k1 --> (continue k1 6)} 
			\kr \tp{(continue (not(-), iszero(-), suma(2, -)), 6)}
			\]
			
			\[
			\tx{k2 --> (continue k2 false)} 
			\kr \tp{(continue (not(-)), false)}
			\]
			
>>>>>>> sandy
            \item ¿A qué se evalúa el programa para \tp{x = 7} y para \tp{x $\neq$ 7}?
     		
     		Evaluamos, considerando \tp{x = 7}

			\begin{align*}
			&\pop{\es}{not(letcc[Bool](k2.iszero(2 + letcc[Integer](k1.3 +} \\
			&\tx{\quad \quad  if(x=pred(8),4 * pred(continue(k1,6)), 5 * suc(continue(k2,False))))))}\\
			&\kr \pop{not(-); \es}{letcc[Bool](k2.iszero(2 + letcc[Integer](k1.3 +} \\
			&\tx{\quad \quad \quad if(x=pred(8),4 * pred(continue(k1,6)), 5 * suc(continue(k2,False)))))}\\
			&\kr \pop{not(-); \es}{iszero(2 + letcc[Integer](k1.3 +} \\
			&\tx{\quad \quad \quad if(x=pred(8),4 * pred(continue(k1,6)), 5 * suc(continue(k2,False)))) [k2:= cont(not(-); \es)]}\\
			&\equiv \quad \quad \pop{not(-); \es}{iszero(2 + letcc[Integer](k1.3 +} \\
			&\tx{\quad \quad \quad if(x=pred(8),4 * pred(continue(k1,6)), 5 * suc(continue(cont(not(-); \es),False))))}\\		
			\end{align*}
			\begin{align*}			
			&\kr \pop{iszero(-); not(-); \es}{2 + letcc[Integer](k1.3 +} \\
			&\tx{\quad \quad \quad if(x=pred(8),4 * pred(continue(k1,6)), 5 * suc(continue(cont(not(-); \es),False)))}\\
			&\kr \tx{suma(-, letcc[Integer](k1.3 + if(x=pred(8),4 * pred(continue(k1,6)), 5 *}\\ &\tx{\quad \quad \quad suc(continue(cont(not(-); \es),False)))); iszero(-); not(-); \es $\quad \succ$ 2}\\
			&\kr \tx{suma(-, letcc[Integer](k1.3 + if(x=pred(8),4 * pred(continue(k1,6)), 5 *}\\ &\tx{\quad \quad \quad suc(continue(cont(not(-); \es),False)))); iszero(-); not(-); \es $\quad \prec$ 2}\\
			&\kr \pop{suma(2,-); iszero(-); not(-); \es}{letcc[Integer](k1.3 +}\\
			&\tx{\quad \quad \quad if(x=pred(8),4 * pred(continue(k1,6)), 5 * suc(continue(cont(not(-); \es),False)))}\\
			&\kr \pop{suma(2,-); iszero(-); not(-); \es}{3 +}\\
			&\tx{\quad \quad \quad if(x=pred(8),4 * pred(continue(k1,6)), 5 * suc(continue(cont(not(-); \es),False))}\\
			&\tx{\quad \quad \quad [k1:= cont(suma(2,-); iszero(-); not(-); \es)]}\\
			& \equiv \quad \quad \pop{suma(2,-); iszero(-); not(-); \es}{3 +}\\
			&\tx{\quad \quad \quad if(x=pred(8),4 * pred(continue(cont(suma(2,-); iszero(-); not(-); \es),6)),}\\
			&\tx{\quad \quad \quad 5 * suc(continue(cont(not(-); \es),False))}\\
			& \kr \tx{suma(-, if(x=pred(8),4 * pred(continue(cont(suma(2,-); iszero(-); not(-); \es),6)),}\\
			&\tx{\quad \quad \quad5 * suc(continue(cont(not(-); \es),False))); suma(2,-); iszero(-); not(-); \es}\\
			&\tx{\quad \quad \quad $\succ$ 3}\\
			& \kr \tx{suma(-, if(x=pred(8),4 * pred(continue(cont(suma(2,-); iszero(-); not(-); \es),6)),}\\
			&\tx{\quad \quad \quad 5 * suc(continue(cont(not(-); \es),False))); suma(2,-); iszero(-); not(-); \es}\\
			&\tx{\quad \quad \quad $\prec$ 3}\\
			&\kr \pop{suma(3,-); suma(2,-); iszero(-); not(-); \es}{if(x=pred(8),4 *}\\
			&\tx{\quad \quad \quad pred(continue(cont(suma(2,-); iszero(-); not(-); \es),6)),}\\
			&\tx{\quad \quad \quad 5 * suc(continue(cont(not(-); \es),False)))}\\
			&\kr \tx{if(-, 4 * pred(continue(cont(suma(2,-); iszero(-); not(-); \es),6)) ,}\\
			&\tx {\quad \quad \quad 5 * suc(continue(cont(not(-); \es),False)))); suma(3,-);}\\
			&\tx{\quad \quad \quad suma(2,-); iszero(-); not(-); \es $\ \succ $ x=pred(8)}\\
			&\kr \tx{=(-,pred(8)); if(-, 4 * pred(continue(cont(suma(2,-); iszero(-); not(-); \es),6)) ,}\\
			&\tx {\quad \quad \quad 5 * suc(continue(cont(not(-); \es),False)))); suma(3,-);}\\
			&\tx{\quad \quad \quad suma(2,-); iszero(-); not(-); \es $\ \succ $ x} \tx{\quad con x=7}\\	
			&\kr \tx{=(-,pred(8)); if(-, 4 * pred(continue(cont(suma(2,-); iszero(-); not(-); \es),6)) ,}\\
			&\tx {\quad \quad \quad 5 * suc(continue(cont(not(-); \es),False)))); suma(3,-);}\\
			&\tx{\quad \quad \quad suma(2,-); iszero(-); not(-); \es $\ \prec $ x} \tx{\quad con x=7}\\	
			&\kr \tx{=(x,-); if(-, 4 * pred(continue(cont(suma(2,-); iszero(-); not(-); \es),6)) ,}\\
			&\tx {\quad \quad \quad 5 * suc(continue(cont(not(-); \es),False)))); suma(3,-);}\\
			&\tx{\quad \quad \quad suma(2,-); iszero(-); not(-); \es $\ \succ $ pred(8)} \\
			&\kr \tx{pred(-); =(x,-); if(-, 4 * pred(continue(cont(suma(2,-); iszero(-); not(-); \es),6)) ,}\\
			&\tx {\quad \quad \quad 5 * suc(continue(cont(not(-); \es),False)))); suma(3,-);}\\
			&\tx{\quad \quad \quad suma(2,-); iszero(-); not(-); \es $\ \succ $ 8} \\
			\end{align*}
			\begin{align*}
			&\kr \tx{pred(-); =(x,-); if(-, 4 * pred(continue(cont(suma(2,-); iszero(-); not(-); \es),6)) ,}\\
			&\tx{\quad \quad \quad 5 * suc(continue(cont(not(-); \es),False)))}\\
			&\tx {\quad \quad \quad 5 * suc(continue(cont(not(-); \es),False)))); suma(3,-);}\\
			&\tx{\quad \quad \quad suma(2,-); iszero(-); not(-); \es $\ \prec $ 8} \\
			&\kr \tx{ =(x,-); if(-, 4 * pred(continue(cont(suma(2,-); iszero(-); not(-); \es),6)) ,}\\
			&\tx{\quad \quad \quad 5 * suc(continue(cont(not(-); \es),False)))}\\
			&\tx {\quad \quad \quad 5 * suc(continue(cont(not(-); \es),False)))); suma(3,-);}\\
			&\tx{\quad \quad \quad suma(2,-); iszero(-); not(-); \es $\ \succ $ 7} \\
			&\kr \tx{ =(x,-); if(-, 4 * pred(continue(cont(suma(2,-); iszero(-); not(-); \es),6)) ,}\\
			&\tx{\quad \quad \quad 5 * suc(continue(cont(not(-); \es),False)))}\\
			&\tx {\quad \quad \quad 5 * suc(continue(cont(not(-); \es),False)))); suma(3,-);}\\
			&\tx{\quad \quad \quad suma(2,-); iszero(-); not(-); \es $\ \prec $ 7} \\
			&\kr \tx{if(-, 4 * pred(continue(cont(suma(2,-); iszero(-); not(-); \es),6)) ,}\\
			&\tx{\quad \quad \quad 5 * suc(continue(cont(not(-); \es),False)))}\\
			&\tx {\quad \quad \quad 5 * suc(continue(cont(not(-); \es),False)))); suma(3,-);}\\
			&\tx{\quad \quad \quad suma(2,-); iszero(-); not(-); \es $\ \prec $ True} \\			
			&\kr \pop{suma(3,-); suma(2,-); iszero(-); not(-); \es} {4 *}\\
			&\tx{\quad \quad \quad pred(continue(cont(suma(2,-); iszero(-); not(-); \es),6))} \\
			&\kr \tx{*(-, pred(continue(cont(suma(2,-); iszero(-); not(-); \es),6)));}\\
			&\tx{\quad \quad \quad suma(3,-); suma(2,-); iszero(-); not(-); \es $\ \succ $ 4}\\
			&\kr \tx{*(-, pred(continue(cont(suma(2,-); iszero(-); not(-); \es),6));}\\
			&\tx{\quad \quad \quad suma(3,-); suma(2,-); iszero(-); not(-); \es $\ \prec $ 4}\\
			&\kr \tx{*(4,-); suma(3,-); suma(2,-); iszero(-); not(-); \es}\\
			&\tx{\quad \quad \quad $\succ $ pred(continue(cont(suma(2,-); iszero(-); not(-); \es),6))}\\
			&\kr \tx{pred(-); *(4,-); suma(3,-); suma(2,-); iszero(-); not(-); \es}\\
			&\tx{\quad \quad \quad $\succ $ continue(cont(suma(2,-); iszero(-); not(-); \es),6)}\\
			&\kr \tx{continue(-,6); pred(-); *(4,-); suma(3,-); suma(2,-); iszero(-); not(-); \es}\\
			&\tx{\quad \quad \quad $\succ $ cont(suma(2,-); iszero(-); not(-); \es)}\\
			&\kr \tx{continue(-,6); pred(-); *(4,-); suma(3,-); suma(2,-); iszero(-); not(-); \es}\\
			&\tx{\quad \quad \quad $\prec $ cont(suma(2,-); iszero(-); not(-); \es)}\\
			&\kr \tx{continue(cont(suma(2,-); iszero(-); not(-); \es),-); pred(-); *(4,-);}\\
			&\tx {\quad \quad \quad suma(3,-); suma(2,-); iszero(-); not(-); \es  $\ \succ $6}\\
			&\kr \tx{continue(cont(suma(2,-); iszero(-); not(-); \es),-); pred(-); *(4,-);}\\
			&\tx {\quad \quad \quad suma(3,-); suma(2,-); iszero(-); not(-); \es  $\ \prec $6}\\
			&\kr \push {suma(2,-); iszero(-); not(-); \es}{6}\\
			&\kr \push {iszero(-); not(-); \es}{8}\\
			&\kr \push {not(-); \es}{False}\\
			&\kr \push {\es}{True}\\
			\end{align*} 

			
			Ahora 	Evaluamos, considerando \tp{x $\neq$ 7}
			
			\begin{align*}
			&\pop{\es}{not(letcc[Bool](k2.iszero(2 + letcc[Integer](k1.3 +} \\
			&\tx{\quad \quad  if(x=pred(8),4 * pred(continue(k1,6)), 5 * suc(continue(k2,False))))))}\\
			&\kr \pop{not(-); \es}{letcc[Bool](k2.iszero(2 + letcc[Integer](k1.3 +} \\
			&\tx{\quad \quad \quad if(x=pred(8),4 * pred(continue(k1,6)), 5 * suc(continue(k2,False)))))}\\
			&\kr \pop{not(-); \es}{iszero(2 + letcc[Integer](k1.3 +} \\
			&\tx{\quad \quad \quad if(x=pred(8),4 * pred(continue(k1,6)), 5 * suc(continue(k2,False)))) [k2:= cont(not(-); \es)]}\\
			&\equiv \quad \quad \pop{not(-); \es}{iszero(2 + letcc[Integer](k1.3 +} \\
			&\tx{\quad \quad \quad if(x=pred(8),4 * pred(continue(k1,6)), 5 * suc(continue(cont(not(-); \es),False))))}\\		
			&\kr \pop{iszero(-); not(-); \es}{2 + letcc[Integer](k1.3 +} \\
			&\tx{\quad \quad \quad if(x=pred(8),4 * pred(continue(k1,6)), 5 * suc(continue(cont(not(-); \es),False)))}\\
			&\kr \tx{suma(-, letcc[Integer](k1.3 + if(x=pred(8),4 * pred(continue(k1,6)), 5 *}\\ &\tx{\quad \quad \quad suc(continue(cont(not(-); \es),False)))); iszero(-); not(-); \es $\quad \succ$ 2}\\
			&\kr \tx{suma(-, letcc[Integer](k1.3 + if(x=pred(8),4 * pred(continue(k1,6)), 5 *}\\ &\tx{\quad \quad \quad suc(continue(cont(not(-); \es),False)))); iszero(-); not(-); \es $\quad \prec$ 2}\\
			&\kr \pop{suma(2,-); iszero(-); not(-); \es}{letcc[Integer](k1.3 +}\\
			&\tx{\quad \quad \quad if(x=pred(8),4 * pred(continue(k1,6)), 5 * suc(continue(cont(not(-); \es),False)))}\\
			&\kr \pop{suma(2,-); iszero(-); not(-); \es}{3 +}\\
			&\tx{\quad \quad \quad if(x=pred(8),4 * pred(continue(k1,6)), 5 * suc(continue(cont(not(-); \es),False))}\\
			&\tx{\quad \quad \quad [k1:= cont(suma(2,-); iszero(-); not(-); \es)]}\\
			& \equiv \quad \quad \pop{suma(2,-); iszero(-); not(-); \es}{3 +}\\
			&\tx{\quad \quad \quad if(x=pred(8),4 * pred(continue(cont(suma(2,-); iszero(-); not(-); \es),6)),}\\
			&\tx{\quad \quad \quad 5 * suc(continue(cont(not(-); \es),False))}\\
			& \kr \tx{suma(-, if(x=pred(8),4 * pred(continue(cont(suma(2,-); iszero(-); not(-); \es),6)),}\\
			&\tx{\quad \quad \quad5 * suc(continue(cont(not(-); \es),False))); suma(2,-); iszero(-); not(-); \es}\\
			&\tx{\quad \quad \quad $\succ$ 3}\\
			& \kr \tx{suma(-, if(x=pred(8),4 * pred(continue(cont(suma(2,-); iszero(-); not(-); \es),6)),}\\
			&\tx{\quad \quad \quad5 * suc(continue(cont(not(-); \es),False))); suma(2,-); iszero(-); not(-); \es}\\
			&\tx{\quad \quad \quad $\prec$ 3}\\
			&\kr \pop{suma(3,-); suma(2,-); iszero(-); not(-); \es}{if(x=pred(8),4 *}\\
			&\tx{\quad \quad \quad pred(continue(cont(suma(2,-); iszero(-); not(-); \es),6)),}\\
			&\tx{\quad \quad \quad 5 * suc(continue(cont(not(-); \es),False)))}\\
			&\kr \tx{if(-, 4 * pred(continue(cont(suma(2,-); iszero(-); not(-); \es),6)) ,}
			&\tx{\quad \quad \quad 5 * suc(continue(cont(not(-); \es),False)))}\\
			&\tx {\quad \quad \quad 5 * suc(continue(cont(not(-); \es),False)))); suma(3,-);}\\
			&\tx{\quad \quad \quad suma(2,-); iszero(-); not(-); \es $\ \succ $ x=pred(8)}\\
			&\kr \tx{=(-,pred(8)); if(-, 4 * pred(continue(cont(suma(2,-); iszero(-); not(-); \es),6)) ,}
			&\tx{\quad \quad \quad 5 * suc(continue(cont(not(-); \es),False)))}\\
			&\tx {\quad \quad \quad 5 * suc(continue(cont(not(-); \es),False)))); suma(3,-);}\\
			&\tx{\quad \quad \quad suma(2,-); iszero(-); not(-); \es $\ \succ $ x} \tx{\quad con x=7}\\	
			&\kr \tx{=(-,pred(8)); if(-, 4 * pred(continue(cont(suma(2,-); iszero(-); not(-); \es),6)) ,}
			&\tx{\quad \quad \quad 5 * suc(continue(cont(not(-); \es),False)))}\\
			&\tx {\quad \quad \quad 5 * suc(continue(cont(not(-); \es),False)))); suma(3,-);}\\
			&\tx{\quad \quad \quad suma(2,-); iszero(-); not(-); \es $\ \prec $ x} \tx{\quad con x $\neq$ 7}\\	
			&\kr \tx{=(x,-); if(-, 4 * pred(continue(cont(suma(2,-); iszero(-); not(-); \es),6)) ,}
			&\tx{\quad \quad \quad 5 * suc(continue(cont(not(-); \es),False)))}\\
			&\tx {\quad \quad \quad 5 * suc(continue(cont(not(-); \es),False)))); suma(3,-);}\\
			&\tx{\quad \quad \quad suma(2,-); iszero(-); not(-); \es $\ \succ $ pred(8)} \\
			&\kr \tx{pred(-); =(x,-); if(-, 4 * pred(continue(cont(suma(2,-); iszero(-); not(-); \es),6)) ,}
			&\tx{\quad \quad \quad 5 * suc(continue(cont(not(-); \es),False)))}\\
			&\tx {\quad \quad \quad 5 * suc(continue(cont(not(-); \es),False)))); suma(3,-);}\\
			&\tx{\quad \quad \quad suma(2,-); iszero(-); not(-); \es $\ \succ $ 8} \\
			\end{align*}
			\begin{align*}
			&\kr \tx{pred(-); =(x,-); if(-, 4 * pred(continue(cont(suma(2,-); iszero(-); not(-); \es),6)) ,}\\
			&\tx{\quad \quad \quad 5 * suc(continue(cont(not(-); \es),False)))}\\
			&\tx {\quad \quad \quad 5 * suc(continue(cont(not(-); \es),False)))); suma(3,-);}\\
			&\tx{\quad \quad \quad suma(2,-); iszero(-); not(-); \es $\ \prec $ 8} \\
			&\kr \tx{ =(x,-); if(-, 4 * pred(continue(cont(suma(2,-); iszero(-); not(-); \es),6)) ,}\\
			&\tx{\quad \quad \quad 5 * suc(continue(cont(not(-); \es),False)))}\\
			&\tx {\quad \quad \quad 5 * suc(continue(cont(not(-); \es),False)))); suma(3,-);}\\
			&\tx{\quad \quad \quad suma(2,-); iszero(-); not(-); \es $\ \succ $ 7} \\
			&\kr \tx{ =(x,-); if(-, 4 * pred(continue(cont(suma(2,-); iszero(-); not(-); \es),6)) ,}\\
			&\tx{\quad \quad \quad 5 * suc(continue(cont(not(-); \es),False)))}\\
			&\tx {\quad \quad \quad 5 * suc(continue(cont(not(-); \es),False)))); suma(3,-);}\\
			&\tx{\quad \quad \quad suma(2,-); iszero(-); not(-); \es $\ \prec $ 7} \\
			&\kr \tx{if(-, 4 * pred(continue(cont(suma(2,-); iszero(-); not(-); \es),6)) ,}\\
			&\tx{\quad \quad \quad 5 * suc(continue(cont(not(-); \es),False)))}\\
			&\tx {\quad \quad \quad 5 * suc(continue(cont(not(-); \es),False)))); suma(3,-);}\\
			&\tx{\quad \quad \quad suma(2,-); iszero(-); not(-); \es $\ \prec $ False} \\
			&\kr \pop{suma(3,-); suma(2,-); iszero(-); not(-); \es} {5 * suc(continue(cont(not(-); \es),False))} \\		
			&\kr \pop{*(-,suc(continue(cont(not(-); \es),False))); suma(3,-); suma(2,-); iszero(-); not(-); \es} {5} \\	
			&\kr \push{*(-,suc(continue(cont(not(-); \es),False))); suma(3,-); suma(2,-); iszero(-); not(-); \es} {5} \\
			&\kr \pop{*(5,); suma(3,-); suma(2,-); iszero(-); not(-); \es} {suc(continue(cont(not(-); \es),False))} \\
			&\kr \pop{suc(-); *(5,-); suma(3,-); suma(2,-); iszero(-); not(-); \es} {continue(cont(not(-); \es),False)} \\	
			&\kr \pop{continue(-,False); suc(-); *(5,-); suma(3,-); suma(2,-);iszero(-);not(-); \es} {cont(not(-);\es)} \\	
			&\kr \push{continue(-,False); suc(-); *(5,-); suma(3,-); suma(2,-);iszero(-);not(-); \es} {cont(not(-);\es)} \\	
			&\kr \pop{continue(cont(not(-); \es),-); suc(-); *(5,-); suma(3,-);suma(2,-);iszero(-); not(-);\es} {False} \\
			&\kr \push{continue(cont(not(-); \es),-); suc(-); *(5,-); suma(3,-);suma(2,-);iszero(-); not(-);\es} {False} \\
			&\kr \push{not(-); \es} {False} \\
			&\kr \push{\es} {True} \\				
			\end{align*} 
        \end{enumerate}

        \item \pt{1.5} Considere la siguiente función \tp{N} que depende de
        ciertas funciones dadas \tp{f, g, h}.

        \begin{minted}{haskell}
N 0 = 17

N 1 = f (1 + 13)

N 2 = 22 + (2-3) + 2

N 3 = 22 + (f 3) + 37

N 4 = g 22 (f 4)

N 4 = 22 + (f 4) + 33 + (g y)

N x = h (f x) (44 - y) (g y)
        \end{minted}

        Defina la versión \tp{cps} de \tp{N}, denotada por \tp{cpsN}. Para esto,
        puede suponer definidas las versiones \tp{cps} de \tp{f, g, h},
        denotadas por \tp{cpsf, cpsg, cpsh}.

        \ti{Atención:} Las operaciones aritméticas deben permanecer sin cambios.
        Es decir, no se piden las versiones \tp{cps} de \tp{+, -}.

		\tb{Solución}
		\begin{minted}{haskell}
cpsN 0 k = 17

cpsN 1 k = cpsf (1 + 13) (\v -> k v)

cpsN 2 k = k (22 + (2-3) + 2)

cpsN 3 k = cpsf 3 (\v -> k (22 + v + 37))

cpsN 4 k = cpsg 22 (\v1 -> cpsf 4 (\v2 -> k(v1 + v2))

cpsN 5 k = cpsf 4 (\v1 -> cpsg y (\v2 -> k(22 + v1 + 33+ v2)))

cpsN x k = cpsf x (\v1 -> cpsg y (\v2 -> cpsh v1 (44 -y) v2 (\v3 -> k v3)))
		\end{minted}

        \item \pt{1.5} Considere las siguientes cuatro caracterísiticas para un
        lenguaje de programación.

        \begin{enumerate}
            \item Existe un tipo de cadena \tp{String} y se cumple
            \tp{Int $\leq$ String} mediante una conversión implícita que
            transforma a un entero en una cadena.

            Por ejemplo, \tp{123} a \tp{"123"}.

            \item Existe un tipo cadena \tp{String} que cumple que
            \tp{String $\leq$ Int} mediante una conversión implícita que
            transforma cadenas a enteros ignorado los caractéres que no sean
            dígitos, excepto por el caracter inicial \tp{-}.

            Por ejemplo, \tp{"}\tp{-0aw23r4"} corresponde a \tp{-234}.

            \item Existe un operador binario \tp{+} que denota a ambas la suma
            de enteros y la concatenación de cadenas.

            \item Existe un operador binario \tp{=} que denota a ambas la
            igualdad de enteros y de cadenas.
        \end{enumerate}

        Para cada par de estas caracterísiticas, discuta si se violan o no los
        principios fundamentales del subtipado. En caso afirmativo escriba un
        ejemplo de un programa simple que cause un comprotamiento ambigüo o
        contrainituitivo.

        \begin{itemize}
            \item \tp{String $\leq$ Int} e \tp{Int $\leq$ String}

            Si se tiene algún operador \tp{*} sobrecargado (como en los demás
            casos), entonces podría existir amigüedad en la operación.

            ¿Cuál de los dos valores debería recibir el casting?

            En una implementación se podría resolver fijando un elemento como el
            tipo por omisión para estos caso, pero significaría que la función
            podría dar diferentes valores en ciertas condiciones dependiendo las
            desiciones de diseño del lenguaje, independientemente de la
            definición de la función.

<<<<<<< HEAD
            Por ejemplo, con una expresión \tp{1+"2"}, ¿cuál es el tipo del
            resultado? Usando únicamente los principios básicos del subtipado, 
            ambos tipos serían correctos, lo que haría que una expresión tenga 
            dos tipos.

            Además, el casting hacia entero provoca pérdida de información en 
            los datos.

            \item \tp{String < Int} y \tp{+} sobrecargado
=======
            \item \tp{String $\leq$ Int} y \tp{+} sobrecargado
>>>>>>> sandy

            Considere

            \begin{minted}{haskell}
("1"+"1")+1 = 12
"1"+("1"+1) = 3
            \end{minted}

            Es es que \tp{+} ya no sería asociativa, lo cuál sí cumpliría si
            \tp{String $\not \leq$ Int}.

            \item \tp{String $\leq$ Int} y \tp{=} sobrecargado

            Considere

            \begin{minted}{haskell}
"1q" = 1
1 = "1e"
"1q" /= "1e"
            \end{minted}

            Es decir, la igualdad ya no es transitiva, lo cuál es chocante
            porque ya no sería una relación de equivalencia.

            \item \tp{Int $\leq$ String} y \tp{+} sobrecargado

            Considere

            \begin{minted}{haskell}
(1+2)+"a" = "3a"
1+(2+"a") = "12a"
            \end{minted}

            Es es que \tp{+} ya no sería asociativa, lo cuál sí cumpliría si
            \tp{Int $\not \leq$ String}.

            Fuera de este inconveniente, todo respecto a los tipos se mantiene
            en regla.

<<<<<<< HEAD
            \item \tp{Int < String} y \tp{=} sobrecargado

            El único caso donde se haría cohersión sería al tener una cadena y 
            un número. Pero la única cadena a la que el número es igual es a su 
            converisón dígito a dígito.

            Por lo que aún cuando se mezclen los argumentos, se mantiene la 
            igualdad.

            No hay ningún aspecto inusual en este caso.
=======
            \item \tp{Int $\leq$ String} y \tp{=} sobrecargado
>>>>>>> sandy
            \item \tp{+} y \tp{=} sobrecargados

            Si suponemos que no se cumple que algún tipo es subtipo del otro,
            entonces toda aplicación de estos operadores siempre resultría en 
            operaciones entre cosas del mismo tipo.

            Por lo que no hay ningún cambio o aspecto contraintuitivo en este 
            caso.
        \end{itemize}

        \item \pt{1} Usando la definición de números naturales en
        \tp{Java Peso Pluma} vista en clase.

        \begin{enumerate}
            \item Agregue un método \tp{pot} y \tp{leq} para las operaciones de
            potencia y el orden \tp{$\leq$}.

            \tb{Solución:}

            \begin{minted}[mathescape]{java}
// Clase de naturales definida en las notas
class Nat extends Object {
    Object p;

    Nat (Object n) { super(); this.p = n;}

    Nat suc() { return new Nat(this); }

    Nat pred() { return (Nat) this.p; }

    Nat suma(Nat n) { return this.pred().sum(n.suc());}

    Nat multi(Nat n) { return n.suma(this.pred().multi(n));}

    /** método 'pot'
    * @param Nat n exponente al que se va a elevar la potencia.
    * @return $this^n$
    */
    Nat pot(Nat n) { return n.flipPot(this) }

    /** método auxiliar para calcular 'pot'
    * @param Nat n base del números del que $this$ es exponente.
    * @return $n^{this}$
    */
    Nat flipPot(Nat n) {return n.multi(this.pred().flipPot(n))}

    /** método 'leq'
    * $this \leq n \iff this \not > n \iff n \not < this$
    * @param Nat n Número a comparar con $this$.
    * @return $this \leq n$
    */
    Boolean leq(Nat n) {return n.lt(this).not();}

    /** método auxiliar para calcular 'leq'
    * $this < n \iff 0 < n - this$
    * @param Nat n Número a comparar con $this$.
    * @return $this < n$
    */
    Boolean lt(Nat n) {return this.pred().lt(n.pred());}

    // método auxiliar para calcular 'not' en Boolean
    Nat inv() {return new Cero(this);}

}

// Clase del Cero definida en las notas
class Cero extends Nat {
    Cero (Object n) {super(n);}

    Nat pred() {return this;}

    Nat suma(Nat n) {return n;}

    Nat multi(Nat n) {return this;}

    /** calcular el caso base del método auxiliar 'flipPot'
    * @param Nat n número del que $this$ es exponente.
    * @return $n^0 = 1 = suc(0) = this.suc()$
    */
    Nat flipPot(Nat n) {return this.suc();}

    /** calcular el caso base del método auxiliar 'lt'
    * $0 < n \iff bool(n) = True$
    * @param Nat n número a comparar
    * @return $this < n$
    */
    Booelan lt(Nat n) {return new Boolean(n);}

    // método auxiliar para calcular 'not' en Boolean
    Nat inv() {return this.suc();}
}
        \end{minted}

        Donde \tp{not} está definido como

        \begin{minted}{java}
class Boolean extends Object {
    ...
    Boolean not() {return new Boolean(this.m.inv());}
}
        \end{minted}

            \item Modele la clase \tp{Boolean} para el manejo de valores
            booleanos
            \begin{itemize}
                \item La clase debe extender de \tp{Object}

                \item El constructor recibirá un objeto de la clase \tp{Nat}
                para definir su valor \tp{0} representa \tp{false} y \tp{1}
                representa \tp{true}.

                \item Debe tener métodos \tp{true} y \tp{false} que regresen una
                instancia de \tp{Boolean} según el caso.
            \end{itemize}

            \tb{Solución:}
            \begin{minted}{java}
class Boolean extends Object {
    Nat m;

    Boolean (Nat n) { super(); this.m = n;}

    Boolean true() { return new Boolean (new Cero(this).suc());}

    Boolean false() { return new Boolean (new Cero(this));}

    Boolean not() { return new Boolean(this.m.inv());}
}
            \end{minted}
        \end{enumerate}

        \item \pt{2} Definina el siguiente lenguaje \tp{MinEAB}

        \begin{minted}{haskell}
e ::= n | true | false | e + e | e < e
        \end{minted}

        en \tp{Java Peso Pluma}.

        Hay que seguir los siguientes pasos

        \begin{enumerate}
            \item Defina una clase \tp{Expr} que incluya los siguientes métodos

            \begin{itemize}
                \item \tp{isAtom} que devuelve \tp{true} si la expresión no
                tiene subexpresiones propias.

                \item \tp{lsub} que devuelve la subexpresión izquierda de una
                expresión no atómica.

                \item \tp{rsub} que devuelve la subexpresión derecah de una
                expresión no atómica.

                \item \tp{eval} que devuelve el valor de la expresión.
            \end{itemize}

            Esta clase debe ser abstracta en el sentido de que no tiene
            atributos y por lo tanto sus métodos no tiene cuerpo. Es una
            interfaz.

            \tb{Solución:}

            \begin{minted}{java}
class Expr extends Object {

    Boolean isAtom() {return error;}

    Expr lsub() {return error;}

    Expr rsub() {return error;}

    Expr eval() {return error;}
}
            \end{minted}

            \item Defina las siguientes clases que implemente a \tp{Expr}

            \begin{itemize}
                \item \tp{NumExpr} que implemente los métodos para manejar
                números.

                \tb{Solución:}

                \begin{minted}{java}
class NumExpr extends Expr {
    Nat v;

    NumExpr(Nat w) {super(); this.v = w;}

    Boolean isAtom() {return  (new Boolean(v)).true();}

    Expr eval() {return this;}
}
                \end{minted}

                \item \tp{BoolExpr} que implemente los métodos para manejar
                booleanos.

                \tb{Solución:}

                \begin{minted}{java}
class BoolExpr extends Expr {
    Boolean v;

    BoolExpr(Boolean v) {super(); this.v = v;}

    Boolean isAtom() {return this.v.true();}

    Expr eval() {return this;}
}
                \end{minted}

                \item \tp{SumExpr} que implemente los métodos para manejar
                sumas.

                \tb{Solución:}

                \begin{minted}{java}
class SumExpr extends Expr {
    Expr i;
    Expr d;

    SumExpr(Expr e1, Expr e2) {super(); this.i = e1; this.d = e2}

    Boolean isAtom() {return (new Booelan(new Cero(this))).false();}

    Expr eval() {return new NumExpr(this.i.eval().v.suma(this.d.eval().v));}
}
                \end{minted}

                \item \tp{LTExpr} que implemente los métodos para manejar
                comparaciones de orden.

                \tb{Solución:}

                \begin{minted}{java}
class LTExpr extends Expr {
    Expr i;
    Expr d;

    LTExpr(Expr e1, Expr e2) {super(); this.i = e1; this.d = e2}

    Boolean isAtom() {return (new Boolean(new Cero(this))).false();}

    Expr eval() {return new BoolExpr(this.i.eval().v.lt(this.d.eval().v));}
}
                \end{minted}

            \end{itemize}

            \item Dé ejemplos de instancias de cada una de las clases
            anteriores.

            \begin{minted}{java}
// Cero
Nat cero = new Cero(this);

// NumExpr
NumExpr ceroE = new NumExpr(cero);

// BoolExpr
BoolExpr falseE = new BoolExpr(new Boolean(cero));

// SumExpr
SumExpr sumCE = new SumExpr(ceroE, ceroE);

// LTExpr
LTExpr lC = new lteXPR(ceroE, ceroE);
            \end{minted}

            \item Extienda \tp{MinEAB} con las expresiones \tp{-e, iszero e}.

            Con esta nueva definición, cree las siguientes clases

            \begin{itemize}
                \item \tp{NegExpr}

                \tb{Solución:}

                Como ahora hay números negativos, hay que definir la clase de
                enteros.

                \begin{minted}[mathescape]{java}
// definición de enteros usando naturales
// parejas de aturales cuya diferencia indica el número a
// representar.
// $n \in \mathbb{N}$
// $n \mapsto (n, 0)$
// $-n \mapsto (0, n)$
class Int extends Object {
    Nat i;
    Nat d;

    // $\mathbb{Z} \cong \mathbb{N}^2$
    Int(Nat i, Nat d) {super(); this.i = i; this.d = d;}

    // $(x, y) +_{\mathbb{Z}} (z, w) = (x+_{\mathbb{N}}y, z+_{\mathbb{N}}w)$
    Int suma(Int n) {return new Int(this.i.suma(n.i), this.d.suma(n.d));}

    // $(x, y) \leq_{\mathbb{Z}} (z, w) \iff xz \leq_{\mathbb{N}} yz$
    Boolean lt(Int n) {return this.i.suma(n.d).lt(n.i.suma(this.d));}

    // $-(a, b) = (b, a)$
    Int neg() {return new Int(this.d, this.i);}
}
                \end{minted}

                Hay que modificar la clase de \tp{NumExpr} para manejar enteros
                \begin{minted}{java}
// modificaciones para manejar enteros
class NumExpr extends Expr {
    Int v;

    NumExpr(Int v) {super(); this.v = v;}
    ...
}
                \end{minted}

                Y con estas nuevas modificaciones se define \tp{NegExpr}
                \begin{minted}{java}
// manejar expresiones con negativos
class NegExpr extends Expr {
    Expr e;

    NegExpr(Expr e) {this.e = e;}

    Boolean isAtom() {return (new Booelan(new Nat(this))).false();}

    Expr eval() {return this.e.eval().neg();}
}
                \end{minted}
                \item \tp{IsZero}

                Para evaluar esto, se modifica la clase \tp{Nat} agregando los
                siguientes método.

                \begin{minted}[mathescape]{java}
// algunas modificaciones para Booleanos
class Boolean extends Object {
    ...
    /**
    * $p \lor q = False \iff p = q = False = Bool(0) = Bool(0+0)$
    */
    Boolean or(Boolean p) {return new Boolean(this.m.suma(p.m));}

    /**
    * $p \land q = \neg(\neg p \lor \neg q)$
    */
    Boolean and(Boolean p) {return this.not().or(p.not()).not();}
}

// modificaciones para Nat
class Nat extends Object {
    ...
    /**
    * $a = b \iff a \leq b \land b \leq a$
    */
    Nat eq(Nat n) {return this.leq(n).and(n.leq(this));}
    ...
}

// modificacioes para Int
class Int extends Object {
    ...
    Booelan isZero(Int i) {return this.i.eq(this.d);}
    ...
}
                \end{minted}

                Y se usa este método para

                \begin{minted}{java}
class IsZero extends Expr {
    Expr e;

    isZero(Expr e) {super(); this.e = e;}

    Boolean isAtom() {return (new Booelan(new Nat(this))).false();}

    Expr eval() {return new BoolExpr(e.eval().v.isZero());}
}
                \end{minted}
            \end{itemize}

            \item Dé ejemplos de instancias de estas dos clases.

            \begin{minted}{java}
// cero
Int ceroI = new Int(new Cero(this), new Cero(this));

// NegExpr
NegExpr ceroE = new NegExpr(ceroI);

// IsZero
IsZero isZE = new IsZero(ceroI);
            \end{minted}

            \item ¿Cómo se modifican las subclases de \tp{Expr} definidas en
            puntos anteriores?

            Para manejar \tp{NegExpr}, definió la clase \tp{Int} y se remplazó
            \tp{Nat} en \tp{NumExpr} por \tp{Int}.

            Para definir \tp{IsZero}, sólo se modificaron las clases de
            \tp{Boolean}, \tp{Nat} y \tp{Int} agregando los métodos necesarios
            para usar \tp{isZero} en \tp{Int}.
        \end{enumerate}

        Puede suponer definida la clase \tp{Value} (escencialmente
        \tp{Nat + Bool}) cuyas instancias sean los valores del lenguaje.

        Además de otras clases primitivas con los métodos que requiera.

        También se puede usar la constantee de error en cualquier método.

        \item \tb{Extra \pt{hasta 2}: Privacidad en \tp{Java Peso Pluma}}

        \tp{Java} proporciona mecanismos para controlar el accesso. Un método o
        atributo de una clase pueden declararse como \ti{público, protegido o
        privado}. En \tp{Java Peso Pluma} es posible agregar este mecanismo como
        sigue

        \begin{minted}[escapeinside=<>]{java}
C ::= class C extends C {<\tvec{p}\tvec{C}\tvec{f}>;K <\tvec{M}>} // declaración de clases
M ::= p C m (<\tvec{C} \tvec{x}>) {return e;} // declaración de métodos
p ::= public | protected | private // modificacor de privacidad
        \end{minted}

        EL significado intuitivo de los modificadores de privacidad es el
        siguiente:

        \begin{itemize}
            \item Si un método o atributo de una clase \tp{C} se declara
            \tp{public}, entonces se permite el acceso desde cualquier lugar.

            \item Si un método o atributo de una clase \tp{C} se declara
            \tp{protected}, entonces se permite el acceso únicamente desde
            métodos de \tp{C} y subclases de \tp{C}.

            \item Si un método o atributo de una clase \tp{C} se declara
            \tp{private}, entonces se permite el acceso únicamente desde
            métodos de \tp{C}.
        \end{itemize}

        Extienda la semática estática para filtrar programas que contengan
        violaciones de privacidad de acuerdo a las reglas dadas arriba.
        Especifique claramente cuales reglas de tipado originales se eliminan o
        se sustituyen por nuevas y cuáles se mantienen.

        \ti{Sugerencia:} Cuando se verifique si un método está bien formado en
        una clase \tp{C}, se debe verificar si la expresión en el cuerpo del
        método no se reifere a métodos o atributos en otra clase \tp{D} que sean
        privados o protegidos si \tp{C $\not \leq$ D}. Podría necesitarse el
        paso de algo más en el contexto \tp{$\Gamma$} en el juicio de tipado
        para expresiones.

    \end{enumerate}
\end{document}
