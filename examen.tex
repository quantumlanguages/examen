% type 
\documentclass{article}

% page format
\usepackage[letterpaper, margin=1.5cm]{geometry}

% font
\renewcommand{\familydefault}{\sfdefault}

% colors
\usepackage[dvipsnames]{xcolor}

% enumeration format
\usepackage{enumitem}
\setlist[enumerate,1]{label=\textcolor{RoyalPurple}{\arabic*.}}
\setlist[enumerate,2]{label=\textcolor{RoyalPurple}{\alph*)}}
\setlist[itemize,1]{label=\color{RoyalPurple}{$\bullet$}}

% language
\usepackage[spanish]{babel}

% math symbols
\usepackage{amssymb}

% code
\usepackage{minted}
\usemintedstyle{manni}

% shortcuts
\newcommand{\tx}[1]{\textcolor{CadetBlue} {\texttt{#1}}}
\newcommand{\tb}[1]{\textcolor{RoyalPurple} {\textbf{#1}}}
\newcommand{\ti}[1]{\textcolor{RoyalPurple} {\textit{#1}}}
\newcommand{\pt}[1]{\textcolor{RoyalPurple}{(#1pt)}}

% header
\title {
    \color{RoyalPurple}
    Lenguajes de Programación 2020-1\\
    Facultad de Ciencias UNAM\\
    Tarea Examen Parcial 4
}

\author{
    Sandra del Mar Soto Corderi \quad Edgar Quiroz Castañeda
}

\date{
    Fecha de entrega: 2 de diciembre de 2019
}

% document
\begin{document}
    % header
    \maketitle

    \tb{Esta tarea vale 8 puntos sobre el parcial 4, la calificación se 
    completa con una pregunta presencial el 29 de noviembre de 2019}

    \begin{enumerate}
        \item \pt{1} Se desea implementar una función \tx{ct} que reciba un 
        árbol heterogéneo de naturales o booleanos y devuelva la conjunción de 
        sus elementos siempre y cuando todos sean booleanos y en otro caso 
        devuelva el valor \tx{n+1}, donde \tx{n} es el primer natural encontrado en el
        árbol. Para este propósito defina una función \tx{ctaux} y una expresión
        \tx{e} tal que la función \tx{ct} quede implementada como
        
        \begin{minted}{haskell}
            ct t = handle ctaux y with x => e
        \end{minted}

        Bosqueje la evaluación de la expresión

        \begin{minted}{haskell}
            Node (iszero 9) (Node False Void Void) (Node 5 Void Void)
        \end{minted}

        en la máquina \tx{$\mathcal{K}$}.

        Puede omitir varios pasos pero no los que involucren el manejo de 
        excepciones.

        \ti{Sugerencia:} Es más fácil si define \tx{ctaux} a partir de una
        función binaria \tx{vand} que realice la conjunción si sus argumentos
        son booleanos y en caso contrario lance una excepción adecuada de forma
        que sea manejada por \tx{e}. Puede suponer que existe una función unaria 
        \tx{isbool} que verifica si su argumento es o no un booleano.

        \item \pt{1} Considere el siguiente programa, donde suponemos que el 
        lenguaje contiene un operador primitivo \tx{not} para la negación 
        booleana.

        \begin{minted}{haskell}
            not letcc k2 in
                iszero(2 + letcc k1 in)
                    3  + if x = pred 8 then 
                        4 * pred (continue k2 6) 
                        else 5 * suc (continue k1 false)
                    end
                    )
                end
            end
        \end{minted}
        \begin{enumerate}
            \item ¿Cuáles son los tipos de \tx{k1} y \tx{k2}?
            \item ¿A qué continuaciones se ligan las variables \tx{k1} y 
            \tx{k2}?
            \item ¿A qué se evalua el programa para \tx{x = 7} y para 
            \tx{x $\neq$ 7}?
        \end{enumerate}

        \item \pt{1.5} Considere la siguiente función \tx{N} que depende de
        ciertas funciones dadas \tx{f, g, h}.

        \begin{minted}{haskell}
            N 0 = 17

            N 1 = f (1 + 13)

            N 2 = 22 + (2-3) + 2

            N 3 = 22 + (f 3) + 37

            N 4 = g 22 (f 4)

            N 4 = 22 + (f 4) + 33 + (g y)

            N x = h (f x) (44 - y) (g y)
        \end{minted}

        Defina la versión \tx{cps} de \tx{N}, denotada por \tx{cpsN}. Para esto,
        puede suponer definidas las versiones \tx{cps} de \tx{f, g, h},
        denotadas por \tx{cpsf, cpsg, cpsh}.

        \ti{Atención:} Las operaciones aritméticas deben permanecer sin cambios.
        Es decir, no se piden las versiones \tx{cps} de \tx{+, -}.

        \item \pt{1.5} Considere las siguientes cuatro caracterísiticas para un 
        lenguaje de programación.

        \begin{enumerate}
            \item Existe un tipo de cadena \tx{String} y se cumple 
            \tx{Int $\leq$ String} mediante una conversión implícita que 
            transforma a un entero en una cadena. 

            Por ejemplo, \tx{123} a \tx{"123"}.

            \item Existe un tipo cadena \tx{String} que cumple que 
            \tx{String $\leq$ Int} mediante una conversión implícita que 
            transforma cadenas a enteros ignorado los caractéres que no sean
            dígitos, excepto por el caracter inicial \tx{-}.

            Por ejemplo, \tx{"}\tx{-0aw23r4"} corresponde a \tx{-234}.

            \item Existe un operador binario \tx{+} que denota a ambas la suma 
            de enteros y la concatenación de cadenas.

            \item Existe un operador binario \tx{=} que denota a ambas la 
            igualdad de enteros y de cadenas.
        \end{enumerate}

        Para cada par de estas caracterísiticas, discuta si se violan o no los 
        principios fundamentales del subtipado. En caso afirmativo escriba un
        ejemplo de un programa simple que cause un comprotamiento ambigüo o 
        contrainituitivo.

        \item \pt{1} Usando la definición de números naturales en 
        \tx{Java Peso Pluma} vista en clase.

        \begin{enumerate}
            \item Agregue un método \tx{pot} y \tx{leq} para las operaciones de
            potencia y el orden \tx{$\leq$}.

            \item Modele la clase \tx{Boolean} para el manejo de valores
            booleanos
            \begin{itemize}
                \item La clase debe extender de \tx{Object}

                \item El constructor recibirá un objeto de la clase \tx{Nat}
                para definir su valor. \tx{0} representa \tx{true} y \tx{1}
                representa \tx{true}.

                \item Debe tener métodos \tx{true} y \tx{false} que regresen una
                instancia de \tx{Boolean} según el caso.
            \end{itemize} 
        \end{enumerate}
        
        \item \pt{2} Definina el siguiente lenguaje \tx{MinEAB}

        \begin{minted}{haskell}
            e ::= n | true | false | e + e | e < e
        \end{minted}

        en \tx{Java Peso Pluma}.

        Hay que seguir los siguientes pasos

        \begin{enumerate}
            \item Defina una clase \tx{Expr} que incluya los siguientes métodos

            \begin{itemize}
                \item \tx{isAtom} que devuelve \tx{true} si la expresión no
                tiene subexpresiones propias.

                \item \tx{lsub} que devuelve la subexpresión izquierda de una
                expresión no atómica.

                \item \tx{rsub} que devuelve la subexpresión derecah de una
                expresión no atómica.

                \item \tx{eval} que devuelve el valor de la expresión.
            \end{itemize}

            Esta clase debe ser abstracta en el sentido de que no tiene 
            atributos y por lo tanto sus métodos no tiene cuerpo. Es una 
            interfaz.

            \item Defina las siguientes clases que implemente a \tx{Expr}
            
            \begin{itemize}
                \item \tx{NumExpr} que implemente los métodos para manejar 
                números.

                \item \tx{BoolExpr} que implemente los métodos para manejar 
                booleanos.

                \item \tx{SumExpr} que implemente los métodos para manejar 
                sumas.

                \item \tx{LTExpr} que implemente los métodos para manejar 
                comparaciones de orden.
            \end{itemize}

            \item Dé ejemplos de instancias de cada una de las clases 
            anteriores.

            \item Extienda \tx{MinEAB} con las expresiones \tx{-e, iszero e}.

            Con esta nueva definición, cree las siguientes clases

            \begin{itemize}
                \item \tx{NegExpr}
                \item \tx{IsZero}
            \end{itemize}

            \item Dé ejemplos de instancias de estas dos clases.

            \item ¿Cómo se modifican las subclases de \tx{Expr} definidas en
            puntos anteriores?
        \end{enumerate}

        Puede suponer definida la clase \tx{Value} (escencialmente 
        \tx{Nat + Bool}) cuyas instancias sean los valores del lenguaje.

        Además de otras clases primitivas con los métodos que requiera.

        También se puede usar la constantee de error en cualquier método.

        \item \tb{\pt{hasta 2} Privacidad en \tx{Java Peso Pluma}}

        \tx{Java} proporciona mecanismos para controlar el accesso. Un método o
        atributo de una clase pueden declararse como \ti{public, protegido o 
        privado}. En \tx{Java Peso Pluma} es posible agregar este mecanismo como
        sigue

        \begin{minted}[escapeinside=||]{haskell}
            C ::= class C extends C {|{$\vec{p}\vec{C}\vec{f}$}|;K|$\vec{M}$|} -- declaración de clases
            M ::= p C m (|$\vec{C} \vec{x}$|) {return e;} -- declaración de métodos
            p ::= public | protected | private -- modificacor de privacidad
        \end{minted}

        EL significado intuitivo de los modificadores de privacidad es el 
        siguiente:

        \begin{itemize}
            \item Si un método o atributo de una clase \tx{C} se declara 
            \tx{public}, entonces se permite el acceso desde cualquier lugar.

            \item Si un método o atributo de una clase \tx{C} se declara 
            \tx{protected}, entonces se permite el acceso únicamente desde
            métodos de \tx{C} y subclases de \tx{C}.

            \item Si un método o atributo de una clase \tx{C} se declara 
            \tx{private}, entonces se permite el acceso únicamente desde
            métodos de \tx{C}.
        \end{itemize}

        Extienda la semática estática para filtrar programas que contengan 
        violaciones de privacidad de acuerdo a las reglas dadas arriba. 
        Especifique claramente cuales reglas de tipado originales se eliminan o 
        se sustituyen por nuevas y cuáles se mantienen.

        \ti{Sugerencia:} Cuando se verifique si un método está bien formado en 
        una clase \tx{C}, se debe verificar si la expresión en el cuerpo del 
        método no se reifere a métodos o atributos en otra clase \tx{D} que sean
        privados o protegidos si \tx{C $\not \leq$ D}. Podría necesitarse el 
        paso de algo más en el contexto \tx{$\Gamma$} en el juicio de tipado 
        para expresiones.

    \end{enumerate}
\end{document}
